% プロジェクト学習中間報告書書式テンプレート(edited t4t5u0) ver.1.0 (utf-8)

% 両面印刷する場合は `openany' を削除する
\documentclass[openany,10pt,report]{ltjsbook}

% 報告書提出用スタイルファイル
%\usepackage[final]{fun_report_t4t5u0}%最終報告書
\usepackage[middle]{style/fun_report_t4t5u0}%中間報告書

% 画像ファイル (EPS, EPDF, PNG) を読み込むために
\usepackage{graphicx}

\def\hissu{\bgroup\color{red}}
\def\endhissu{\egroup}

% 年度の指定
\thisYear{2021}

% プロジェクト名
\jProjectName{函館版おいしいカレーの作り方プロジェクト}

% [簡易版のプロジェクト名]{正式なプロジェクト名}
% 欧文のプロジェクト名が極端に長い(2行を超える)場合は、短い記述を
% 任意引数として渡す。
%\eProjectName[Making Delicious curry]{How to make delicious curry of Hakodate}
\eProjectName{How to make delicious Hakodate curry}

% <プロジェクト番号>-<グループ名>
\ProjectNumber{25-D}

% グループ名
\jGroupName{グループ~1}
\eGroupName{Group~1}

% プロジェクトリーダ      注意:学籍番号不用
\ProjectLeader{未来花子}{Hanako~Mirai}

% グループリーダ      注意:学籍番号不用
\GroupLeader  {未来太郎}{Taro~Mirai}

% メンバー数
\SumOfMembers{9}
% グループメンバ      注意:学籍番号不用
\GroupMember  {1}{函館太郎}{Taro~Hakodate}
\GroupMember  {2}{函館花子}{Hanako~Hakodate}
\GroupMember  {3}{北海太郎}{Taro~Hokkai}
\GroupMember  {4}{北海花子}{Hanako~Hokkai}
\GroupMember  {5}{未来花子}{Hanako~Mirai}
\GroupMember  {6}{未来一郎}{Ichiro~Mirai}
\GroupMember  {7}{未来二朗}{Jiro~Mirai}
\GroupMember  {8}{未来三郎}{Saburo~Mirai}
\GroupMember  {9}{未来敬祐}{Keisuke~Mirai}

% 指導教員
\jadvisor{指導刷蔵,指導刷子,副出刷子,福田刷男}
% 複数人数いる場合はカンマ(,)で区切る。カンマの前後に空白は入れない。
\eadvisor{Suruo~Shido,Suruko~Shido,Suruko~Hukude,Suruo~Hukude}

% 論文提出日
\jdate{2021年6月28日}
\edate{June~28, 2021}

\begin{document}
%
% 表紙
\maketitle

%======================================================================
%前付け
\frontmatter

% 和文概要
\begin{jabstract} 日本語の概要を書く。
% 和文キーワード
\begin{jkeyword}
キーワード1, キーワード2, キーワード3, キーワード4, キーワード5
\end{jkeyword}
\bunseki{未来太郎}
\end{jabstract}

%英語の概要
\begin{eabstract} Abstract in English. 
% 英文キーワード
\begin{ekeyword}
Keyrods1, Keyword2, Keyword3, Keyword4, Keyword5
\end{ekeyword}
\bunseki{函館花子}
\end{eabstract}

\tableofcontents% 目次

%======================================================================
\mainmatter% 本文のはじまり

%各章の.texファイルをここに並べる
% %ファイル名の「.tex」は省略して良い
% \include{background}
% \include{goal}
% \include{process_overview}
% \include{process_detail}
% \include{result}


% 以降、付録(付属資料)であることを示す
\begin{appendix}

\chapter{新規習得技術}
%課題解決過程に習得した技術について解説する。

\chapter{活用した講義}
%課題解決過程において活用した講義について、講義名・活用内容を記述する。 

\chapter{相互評価}
%課題解決過程で分担し、連携した作業全般について、互いに客観的に評価する。 

\chapter{その他製作物}
%その他成果物をプロジェクトの担当教員の指示に従って添付する。

%付録の終わり
\end{appendix}

%======================================================================
%\backmatter

% 参考文献 bibtex 形式で別ファイルに記述
\bibliography{style/reference}
\bibliographystyle{junsrt}

\end{document}